\documentclass{resume}

\usepackage[left=0.75in,top=0.6in,right=0.75in,bottom=0.6in]{geometry}
\usepackage[utf8]{inputenc}
\usepackage{hyperref}

\name{Marlus Cadanus da Costa}
\address{ Nilo Peçanha, 3803 \\ Curitiba, PR - 82120-440 - Brésil  }
\address{ skype: marluscadanus \\ www.linkedin.com/in/marluscadanus \\ marluscadanus@gmail.com }

\def\nameskip{\bigskip}
\def\sectionskip{\medskip}

\begin{document}

  \begin{rSection}{Sommaire}
     \item Ingénieur de Logiciels détenant 10+ années d’expérience.
     \item Excellente connaissance de la programmation orientée objets avec C, C++, et Java.
     \item Riche expérience dans le domaine de développement des logiciels multiplateforme: Windows, Linux et Web.
     \item Capable d'apprendre et de s'adapter facilement.
     \item Français (avancé), Anglais (avancé) et Portugais (courant).
  \end{rSection}

  \begin{rSection}{Formation}
    {\bf Spécialisation en Informatique - Programmeur de Jeux Vidéo } \hfill {\em 2012} \\
    {PUC-PR, Curitiba - PR - Brésil} \\ 

    {\bf Baccalauréat en Informatique } \hfill {\em 2008} \\
    {Université Positivo, Curitiba - PR - Brésil}   
  \end{rSection}
  
  \begin{rSection}{Technologies Maitrisées}
    \begin{rSubsection}{}{}{}{}
      \item {\bf Concepts:} Design Patterns, OOP, UML, RUP, Scrum, Agile, SOA, BPM, TDD, BDD.
      \item {\bf Langages de programmation:} C, C++ et Java.
      \item {\bf Frameworks/APIs - Java EE:} Servlets, Spring, EJB, MDB, JMS, JPA, JDBC, Web Services (SOAP et REST).
      \item {\bf Frameworks/APIs/Utils - C:} C standard library, Sockets, p\_threads, CMake, GDB, Valgrind.
      \item {\bf Frameworks/APIs/Utils - C++:} Qt, Boost, STD, STL.
      \item {\bf Bases de données}: MySQL, PostgreSQL, Oracle.
      \item {\bf Source Control Management Systems}: SVN, Git.
      \item {\bf Integrated Development Environments}: Qt Creator, XCode, Eclipse et Visual Studio.
      \item {\bf Modeling et Design}: Enterprise Architect, Bizagi, IBM Rational Software Architect.
    \end{rSubsection}
  \end{rSection}
  \begin{rSection}{Expérience Profissionnelle}
    \begin{rSubsection}{\bf Architecte et programmeur}{Déc. 2010 - Fev. 2015}{Winterlabs}{}
    La compagnie Winterlabs est une petite entrepise de development des systemes. Elle
détient de bureau à Curitiba au Brésil. J'ai travaillé comme architecte de solutions et programmeur, mon travail était la conception, le développement et la recherche de meilleures solutions pour les applications d'entreprise.
    \end{rSubsection}
    \begin{rSubsection}{Responsabilités:}{}{}{}
      \item Crée l’architecture des systemes;
      \item Analysé, programmé et testé le code en C/C++/Objective-C et Java;
      \item Effectué le déploiement, la gestion et le support de la solution;
      \item Évalué et testé nouvelles tecnologies;
    \end{rSubsection}

    \begin{rSubsection}{Projets significatifs:}{}{}{}
      \item Projet STAP - Mis en œuvre un simulateur de tir virtuel pour les forces armes du Bresil, développé le module qui contrôle la piste de tir utilisant \textbf{Java}, utilisé \textbf{C/C++} avec OpenGL pour rendre les scènes du stand de tir, utilisé \textbf{Jasper Reports} pour développer les rapports sur les sessions de tir et utilisé \textbf{MySQL} comme base de données.
      \item Projet SKY FIGHTERS - Développé un jeu de carte et stratégie pour \textbf{iOS} en utilisant \textbf{Objective-C} et cocos2d framework.
      \item Projet RESTA1 - Développé un jeux de plateau classique pour \textbf{iOS} en utilisant \textbf{Objective-C} et cocos2d framework.
    \end{rSubsection}

    \begin{rSubsection}{\bf Analyste et programmeur}{Août 2008 - Déc. 2010}{GVT Telecom}{}
    %\item Projet "Nova Arquitetura OSS” – Modification de nouveaux équipements pour augmenter la vitesse du produit ADSL. J'ai développé une nouvelle interface de communication avec le nouvel équipement ADSL, j'ai utilisé \textbf{JEE 1.5} avec \textbf{Oracle 10g} et le serveur \textbf{BEA Weblogic application server}, \textbf{BEA Aqualogic Service Bus}, \textbf{EJB3}, \textbf{JPA}, \textbf{JAX-WS 2.0}, \textbf{JAXB} et \textbf{Eclipse}.\\
    %\item Projet PD “Modificação Portal Integra” – Modification pour fournir de nouvelles informations à la clientèle. J'ai développé l'interface  utilisant des \textbf{portlets} avec \textbf{BPM} et l'architecture \textbf{SOA} utilisant \textbf{Savvion}, j'ai utilisé le serveur d'applications \textbf{BEA Weblogic Portal}, \textbf{BEA Aqualogic Service Bus}.\\
    % \item Projet “PJ 759 – Nova Regulamentação de Call Centers”. Projet pour répondre à la nouvelle réglementation des centres d'appel. J‘ai spécifié le IT Solution avec \textbf{Enterprise Architec} et j'ai développé les \textbf{Stored Procedures} et \textbf{Triggers} en \textbf{PL/SQL}, \textbf{Oracle} comme \textbf{SGBDR} et \textbf{JSP}, \textbf{JSTL}, \textbf{Struts}, \textbf{EJB 2.1} sous \textbf{Oracle IAS} serveur d'application.\\
    % \item Projet “PJ 679 - TT-Retail”. Projet avec l’objectif de manoeuvre des stations de base existantes. J’ai utilisé \textbf{BEA WebLogic Workshop 8}, \textbf{BEA AquaLogic Service Bus}, \textbf{WebLogic Integration (WLI)}, \textbf{JMS}, \textbf{Oracle AQ}, \textbf{MDB} et \textbf{BEA Weblogic Application Server 8}.\\
    %\item Projet PJ 771 - WiseTools. Web services qui exposent les données contenues de la base du TBS/OSS. J’ai utilisé \textbf{JEE 1.5} avec \textbf{Oracle 10g} et \textbf{BEA Weblogic serveur d'application}, \textbf{BEA AquaLogic Service Bus}, \textbf{EJB3}, \textbf{JPA}, \textbf{JAX-WS 2.0}, \textbf{JAXB}, \textbf{Eclipse}, \textbf{Mylyn}. J'ai utilisé une méthode de developpement agile (\textbf{SCRUM}) avec l'outil \textbf{JIRA} et \textbf{Green Hopper} (agile plugin pour burndown chart).\\
    %\item Projet “PJ 702 – Serviços de Gerência”. C’est un système SLA pour gérer les données et la qualité du transfert de voix des clients directs. J’ai utilisé les technologies \textbf{Servlet}, \textbf{JSP}, \textbf{JSTL}, \textbf{AJAX}, \textbf{EJB-doclet}, \textbf{EJB 2.1}, \textbf{Spring}, \textbf{JMS}, \textbf{MDB}, \textbf{Web Services}, \textbf{JasperReports}, \textbf{JFreeChart}, \textbf{ANT} et les design patterns, \textbf{Session Facade}, \textbf{DTO}, \textbf{Business Object}, \textbf{Service Locator}. J'ai utilisé le \textbf{BEA WebLogic Workshop 9.2} comme IDE, le \textbf{BEA Weblogic 9.2} comme serveur d'applications et \textbf{BEA AquaLogic Service Bus} comme \textbf{ESB}.\\
    %\item Projet “Arquitetura Integração Contínua OSS” – Definition de l’architecture de développement d’intégration continue pour les projets de l'équipe OSS-GVT. J'ai utilisé les technologies: \textbf{SVN} pour contrôle de code source, \textbf{Maven} pour compiler le code source, le programme \textbf{NEXUS} pour contrôle d'artefacts, \textbf{BAMBOO} pour gérer et planifier les actions de code source, \textbf{JIRA} comme Issue Tracker, \textbf{JUnit} pour tester le code, \textbf{Selenium} pour faire des tests unitaires automatiques et les plugins \textbf{Checkstyle}, \textbf{PMD}, \textbf{STATSVN}, \textbf{Cobertura}, \textbf{FindBugs} pour faire des rapports sur la qualité du code.
      \begin{rSubsection}{Responsabilités:}{}{}{}
        \item Crée l’architecture des systemes;
        \item Analysé, programmé et testé le code en C/C++/Objective-C et Java;
        \item Effectué le déploiement, la gestion et le support de la solution;
        \item Évalué et testé nouvelles tecnologies;
      \end{rSubsection}

      \begin{rSubsection}{Projets significatifs:}{}{}{}

      \end{rSubsection}
    \end{rSubsection}
    \begin{rSubsection}{HSBC Banque}{Janv. 2006 - Mai. 2008}{Analyste et programmeur}{Curitiba, PR}
    \item Projet “HOB-PWS” - Système développé pour les clients sans compte bancaire; modules Intranet et Internet. J’ai utilisé (Intranet): pattern \textbf{MVC} (\textbf{ITA Framework} / \textbf{MVC2}), intégration avec \textbf{CICS} \textbf{IBM MQSeries} les technologies (Internet): \textbf{Servlets}, \textbf{XML} et \textbf{XSL}. \textbf{WSAD}, \textbf{Websphere 5.1}.\\
    % \item Projet “CDCI”. Système de simulation de financement direct pour le client. J’ai fait la spécification de l’application utilisant le \textbf{IBM RSA}. J’ai développé dans le modèle \textbf{MVC} avec \textbf{Eclipse-RCP}, \textbf{SWT}, la sérialisation d'objets comme couche de sauvegarde.\\
    % \item Projet “HOB-PG-TIT”. Système de paiement de titres utilisés par les clients qui accèdent au site internet de la banque HSBC. Je l’ai développé avec \textbf{Servlets}, \textbf{XML} et \textbf{XSL}, Architecture \textbf{SOA} avec le \textbf{BPM} utilisant \textbf{RTP} (processeur de transactions à distance), l'intégration avec l'héritage (\textbf{CICS}), \textbf{IBM MQSeries}, \textbf{WSAD} (WebSphere Application Developer), \textbf{WebSphere 5.1}.\\
    %\item Projet “GRH - Système de ressources humaines. Je l’ai développé avec le modèle \textbf{MVC} (\textbf{Framework ITA} / \textbf{MVC2}) avec \textbf{JDBC} et \textbf{Sybase}. \textbf{WSAD}; \textbf{Websphere 5.1}
    \end{rSubsection}
  \end{rSection}

  \begin{rSection}{Certifications}
    {\bf Sun Microsystems, Inc., Curitiba - PR - Brésil} \hfill {\em Mai 2005} \\ 
    {Sun Certified Programmer for the Java 2 Platform 1.4} \\

    {\bf Sun Microsystems, Inc., Curitiba - PR - Brésil} \hfill {\em Mai 2008} \\ 
    {Sun Certified Web Component Developer for the Java 2 Platform Enterprise Edition 1.5} \\
  \end{rSection}
  
\end{document}
